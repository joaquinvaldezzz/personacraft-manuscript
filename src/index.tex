\documentclass[conference]{IEEEtran}
\IEEEoverridecommandlockouts
% The preceding line is only needed to identify funding in the first footnote. If that is unneeded, please comment it out.
\usepackage{cite}
\usepackage{amsmath, amssymb, amsfonts}
\usepackage{algorithmic}
\usepackage{graphicx}
\usepackage{textcomp}
\usepackage{xcolor}
\usepackage{array}

\def\BibTeX{\textrm{B\kern-.05em\textsc{i\kern-.025em b}\kern-.08em T\kern-.1667em\lower.7ex\hbox{E}\kern-.125emX}}

\begin{document}
    \title{PersonaCraft: Customer Personality Analysis with Python NLP}

    \author{\IEEEauthorblockN{Elmer B. Delosendo$^{1}$, Jann Kyle O. Adanza$^{2}$, John Joaquin A. Valdez$^{3}$, \\Jon Mark G. Aranas$^{4}$, and Venz Eric Dapadap$^{5}$}
    \IEEEauthorblockA{\textit{College of Computer Studies}\\\textit{Our Lady of Fatima University}\\Quezon City, Metro Manila\\\\Email: $^{1}$ebdelosendo@student.fatima.edu.ph, $^{2}$joadanza7096qc@student.fatima.edu.ph, $^{3}$javaldez1642qc@student.fatima.edu.ph,\\ $^{4}$jgaranas1938qc@student.fatima.edu.ph, $^{5}$vdapadap@student.fatima.edu.ph}}

    \maketitle

    \begin{abstract}
        This case study explores the intriguing field of employing Python Natural Language Processing
        (NLP) tools to analyze human psychology. This study is to investigate the relationship
        between linguistic patterns in written text and different personality attributes by
        utilizing NLP tools. This study uses sentiment analysis, text categorization, and other NLP algorithms
        to look for patterns that may indicate characteristics such as neuroticism, agreeableness, conscientiousness,
        extroversion, and openness. The results of this work have significance not only for the
        field of computational linguistics but also for human-computer interaction, mental health evaluations,
        and personalized recommendation systems. This case study attempts to shed light on the complex
        relationship between language use and human personality traits by utilizing Python's power for
        NLP analysis.
    \end{abstract}

    \begin{IEEEkeywords}
        Python, Natural Language Processing, NLP tools, human psychology, linguistic patterns,
        written text, personality attributes, sentiment analysis, text categorization, neuroticism, agreeableness,
        conscientiousness, extroversion, openness, computational linguistics, human-computer
        interaction, mental health evaluations, personalized recommendation systems, language use,
        personality traits, NLP analysis
    \end{IEEEkeywords}

    \section{Introduction}
    Analyzing and testing a person's personality is crucial to their overall growth. The most
    popular personality test that individuals use is the Myer-Briggs Test Indicator. Even though psychologists
    administer these tests, it is incredibly simple to trick them into giving us the personality
    type we want because the questions are so simple. This research focuses on using images instead of
    questions to automate this task using Neural Networks. For analysis, a tagged dataset containing
    user answers on social media and their personality type is employed. Following cleaning, appropriate
    classification techniques are used to extract pertinent answer attributes using natural language
    processing (NLP) \cite{b1}.

    The advancement of technology has had a major impact on psychological data analysis and forecasting.
    Recent developments in computer technology and the Fourth Industrial Revolution have made it
    possible to quickly and accurately explore and predict human traits, with ongoing advances. In
    contrast to other industries, such as marketing, engineering, and medicine, psychology has been
    slower to incorporate technology, despite these industries rapidly integrating computer expertise
    to generate useful technologies. The investigation and determination of a person's psychological
    condition or characteristics are continuing endeavors. However, because personality is made up
    of complex structures and a wide range of theories, research and technological advancements are hampered
    \cite{b2}.

    This uses the machine learning technique of logistic regression to classify personalities.
    Furthermore, machines capable of Natural Language Processing (NLP) can converse and comprehend in
    human language. The goal of earlier research projects was to automatically determine a person's
    personality type. One important application that offers several benefits is the use of machine learning
    algorithms to classify individuals according to their personalities. In the modern world, knowing
    one's personality can be quite helpful in deciding on a job and personal interests. Personality
    evaluations are a common tool used by modern businesses to streamline the hiring process and
    match candidates with positions that best fit their skills \cite{b3}.

    Using Machine Learning to Predict Textual Data-Based Personality: Personality is a crucial aspect
    of identity as it allows for individual differentiation. Personality prediction is becoming a
    more popular academic topic. Since it reduces time and increases authority by doing away with the
    requirement for customers to fill out surveys, using statistics to predict personality from
    social media is a potential approach. Studying personality is an immensely interesting subject for
    academics. In the real world, personality prediction has many uses. More people utilize social
    media platforms every day. Large volumes of fresh and textual content are posted online every day.
    Current research focuses on Linear Discriminant Analysis, Naive Bayes Multinomial, and AdaBoost
    beyond Twitter norms \cite{b4}.

    \section{Related Works}
    This section explores recent advancements in personality assessment research and related
    methodologies across various domains.

    Pradhan et al. (2020) emphasize the significance of personality testing in individual development,
    highlighting the widespread use of the Myers-Briggs Type Indicator (MBTI). They address concerns
    regarding potential manipulation of traditional personality tests and propose an automated approach
    using Neural Networks and visuals instead of text-based queries. Their research utilizes a
    labeled dataset comprising user comments from social media, coupled with personality types. By
    employing Natural Language Processing (NLP) techniques and classification algorithms, they develop
    a predictive model to accurately detect personality types based on user responses. This model
    serves as the foundation for an interactive personality assessment tool hosted on a website.

    Jang et al. (2022) discuss the transformative impact of technological progress on data analysis and
    forecasting in psychology, particularly in the context of the Fourth Industrial Revolution. Despite
    advancements in other fields, the integration of technology in psychology has been comparatively
    slower. They highlight ongoing efforts to explore and predict psychological states or traits,
    identifying challenges stemming from the complexity of personality theories and structures.

    Chincholkar et al. (2023) introduce the logistic regression machine learning method for personality
    classification, emphasizing the benefits of leveraging NLP-enabled machines for understanding
    and engaging in human language. They underscore the practical applications of personality classification
    in various domains, including career guidance and organizational productivity.

    Archith et al. (2022) discuss the growing academic interest in using machine learning to predict
    personality from textual data, particularly from social media. They highlight the advantages of this
    approach, such as eliminating the need for traditional questionnaires and leveraging the vast
    amount of textual information available online. Their research explores techniques such as Linear
    Discriminant Analysis, Naive Bayes Multinomial, and AdaBoost for personality prediction beyond conventional
    norms observed on platforms like Twitter.

    Together, these studies underscore the evolving landscape of personality assessment and
    prediction, fueled by advancements in technology and data analysis methodologies.

    \section{Methods}
    This chapter outlines how researchers created PersonaCraft: Customer Personality Analysis. The researchers
    used Agile methodology and Python NLP algorithm to develop a system that can analyze customer personality
    from text. Merging Agile's iterative approach with the Python NLP algorithm, they investigate subtle
    personality traits.

    \begin{figure}[ht]
        \centering
        \includegraphics[width=0.75\linewidth]{Manuscript Images/Agile method.png}
        \caption{Agile Development}
    \end{figure}

    \subsection{Planning Phase}

    In this stage, researchers create a road map to develop the system. The researchers investigate the
    works already done on personality and NLP and develop a plan. to decode words in terms of
    personality traits. Researchers make plans for data collection, tool selection, and testing consistency.

    \begin{enumerate}
        \item \textit{Project Framework:} PersonaCraft: Customer Personality Analysis with Python NLP.
            occurs in a few phases that follow, with data collection, followed by preprocessing to
            standardize formats, after that the structured data sets are used to train machine learning
            to differentiate language patterns that correlates with different personalities. While
            validation ensures accuracy the system performance is evaluated through analysis and
            interpretation. This methodical approach aims to create a reliable framework for PersonaCraft:
            Customer Personality Analysis with Python NLP.

            \begin{figure}[ht]
                \centering
                \includegraphics[width=0.75\linewidth]{Manuscript Images/Framework.png}
                \caption{Project Framework}
            \end{figure}

        \item \textit{Architecture of PersonaCraft: Customer Personality Analysis with Python NLP:}
            The overall process of the PersonaCraft: Customer Personality Analysis with Python NLP is
            represented below. The input-process-output is illustrated as follows:

            \begin{figure}[ht]
                \centering
                \includegraphics[width=0.75\linewidth]{Manuscript Images/Input-Process-Output.png}
                \caption{Architecture}
            \end{figure}

            \begin{itemize}
                \item \textit{Input}. Consists of many textual datasets that represent different communication
                    such as service chats, emails, and more.

                \item \textit{Process}. During processing, the system will analyze the input data. The
                    data will undergo preprocessing, analyzing the data, and interpreting the result.
                    Eventually, validation will ensure accuracy. Lastly, the system will classify the
                    personality behind the text.

                \item \textit{Output}. In this phase, the output will show the classified
                    personality behind the input data that the user wrote.
            \end{itemize}

        \item \textit{User Case Modeling:} During this phase, use case modeling illustrates the interaction
            between the system and the user, the objective of a use case modeling is to simplify
            system functions and guide developers to create systems that meet users needs.

        \item \textit{Acceptance Test Criteria:} The system shall allow the user to input any text
            in the system. This may be prepared in the way shown below using the scenario oriented acceptance
            criteria technique.
    \end{enumerate}

    \subsection{Design Phase}

    In this phase, the system's architecture is carefully designed. The researchers opt for a Python
    NLP algorithm implemented using the Flask framework for backend development. For front-end
    development and database management, they utilize the MERN stack, comprising MongoDB, Express,
    React, and Node.js. The system's design is meticulously crafted to optimize performance and
    enhance user experience.

    \subsection{Coding Phase}

    In this phase, the system's construction is elaborated upon. The technical expert employs Visual
    Studio Code as the primary text editor and utilizes the Flask framework for backend development,
    alongside MongoDB and Express for database management. For the frontend, Next.js and Tailwind
    CSS are employed to craft a user-friendly interface.

    \begin{figure}[ht]
        \centering
        \includegraphics[width=1\linewidth]{
            Manuscript Images/CleanShot 2024-05-07 at 18.32.27@2x.png
        }
        \caption{System development done in Visual Studio Code}
    \end{figure}

    \subsection{Testing Phase}

    During this phase, researchers meticulously scrutinize the system to ensure it aligns with the specified
    requirements and standards. A series of tests are conducted to evaluate different facets of the
    system's accuracy. Any discrepancies uncovered during testing are logged, addressed, and subsequently
    retested until the system attains the desired quality benchmarks. Finally, researchers verify the
    system's reliability, accuracy, and readiness for deployment.

    \subsection{Release of the System}

    After completing the testing phase, the system moves forward to the release stage. This entails
    finalizing documentation, crafting user guides, and preparing the system for deployment. Release
    planning encompasses coordinating with end-users, scheduling deployment, and communicating updates
    about the system. Following the release, support and maintenance activities persist to uphold the
    system's integrity and address any post-deployment issues that may arise.

    \begin{table}
        \begin{tabular}{p{4cm}p{4cm}}
            \textbf{Risks}                           & \textbf{Mitigation Measures}                                                       \\
            \hline
            System incompatibility with the hardware & Install and upgrade necessary software, especially the use of the system software. \\
            \hline
            Inevitable deletion of the system        & Provides a backup system.                                                          \\
            \hline
            Not accurate classification              & Train the system countless times to be more accurate.                              \\
        \end{tabular}
        \caption{Risk Assessment and Mitigation Measures}
    \end{table}

    \begin{enumerate}
        \item \textit{Sustainability Measures:} To ensure sustainability measures, continuous support
            and maintenance are required to resolve upcoming bugs or issues. The researchers will
            engage in communication with users where the system is deployed. Additionally, the
            researchers will ask feedback for system growth and relevance for upcoming updates.

        \item \textit{Risk Assessment and Mitigation Measures:} During deployment or upgrading the
            system there might be risks encountered. This will help the researchers to resolve the problem
            and address them before affecting the system.. The system offered a set of testing procedures
            to verify that application deployment and updates will run smoothly and with little issues.
    \end{enumerate}

    \begin{thebibliography}{00}
        \bibitem{b1} T. Pradhan, R. Bhansali, D. Chandnani and A. Pangaonkar, "Analysis of Personality
            Traits using Natural Language Processing and Deep Learning," 2020 Second International Conference
            on Inventive Research in Computing Applications (ICIRCA), Coimbatore, India, 2020, pp.
            457-461, doi: 10.1109/ICIRCA48905.2020.9183090.

        \bibitem{b2} J. Jang, S. Yoon, G. Son, M. Kang, J. Y. Choeh, and K.-H. Choi, "Predicting
            Personality and Psychological Distress Using Natural Language Processing: A Study Protocol,"
            Frontiers in Psychology, vol. 13, Apr. 2022, doi: https://doi.org/10.3389/fpsyg.2022.865541.

        \bibitem{b3} Prof. Amol Chincholkar, Dipti Bhosale, Shivanjali Adsul, Anjali Bodkhe, and R. Kadam,
            "A Comprehensive Survey on Personality Prediction Using Machine Learning Techniques," International
            journal of advanced research in computer and communication engineering, vol. 12, no. 11,
            Nov. 2023, doi: https://doi.org/10.17148/ijarcce.2023.121120.

        \bibitem{b4} P. Archith. "ANALYSIS OF HUMAN TRAITS USING MACHINE LEARNING." Available: https://www.jetir.org/papers/JETIR2205606.pdf
    \end{thebibliography}
    \vspace{12pt}
\end{document}